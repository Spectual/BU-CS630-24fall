\documentclass[11pt]{exam}
\title{\Large{\textbf{CS 630, Fall 2024, Homework 5}}}
\author{\textbf{Yifei Bao}}
\date{\textbf{November 6, 2024}}

\usepackage{fullpage}
\usepackage{xcolor}
\usepackage{amsmath,amsfonts,latexsym,amssymb}
\usepackage{enumerate}
\usepackage[hidelinks]{hyperref}
\usepackage{tikz}
\usepackage{mathtools}
\usepackage{etoolbox}
\usepackage{tipa}
\usepackage{caption}
\usepackage[boxruled,vlined,linesnumbered]{algorithm2e}
\usepackage{enumitem}

\pdfpagewidth 8.5in
\pdfpageheight 11in 

\oddsidemargin 0in
\evensidemargin 0in

\newtheorem{problem}{Problem}

\let\oldtcc\tcc
\renewcommand{\tcc}[1]{{\color{olive}\oldtcc{#1}}}

\begin{document}
\maketitle

% Problem 1
\begin{problem}
    \textbf{Fair numbers from biased coins} 
\end{problem} 

\noindent\textbf{1.}
The algorithm for 2-WayFair($p$) involves flipping the biased coin twice. If the result is “heads-tails” or “tails-heads,” we return 0 or 1, each with an equal probability. If the result is “heads-heads” or “tails-tails,” we disregard the result and try again.\\
The probability of successfully obtaining either “heads-tails” or “tails-heads” is  $2p(1 - p)$ .\\
Therefore, the expected number of iterations it takes for 2-WayFair($p$) to return an answer is $\frac{1}{2p(1 - p)}$.\\

\noindent\textbf{2.}
We call 2-WayFair(p) twice to generate a two-bit binary number  $b_1 b_2$, the result $b_1 b_2$ as folows: 00=0, 01=1, 10=2. Since each of  $b_1 b_2$ = 00 ,  01 , and  10  has an equal probability of  $\frac{1}{4}$ , each outcome (0, 1, 2) is returned with probability  $\frac{1}{3}$\\

Each iteration (two calls to 2-WayFair(p)) is successful with a probability of  $\frac{3}{4}$  (since three out of four two-bit combinations are valid).
Thus, the expected number of iterations for 3-WayFair(p) is:
$\text{Expected iterations} = \frac{4}{3}$


% Algorithm 1.1
\begin{algorithm}[H]
    \caption{3-WayFair$(p)$}
    \While{true}{
        bit1 $\gets$ 2-WayFair($p$)\;
        bit2 $\gets$ 2-WayFair($p$)\;
        
        \tcc{Combine two bits into a binary number}
        result = 2 * bit1 + bit2  \tcc{Forms 00, 01, or 10 as decimal values 0, 1, 2}
        
        \If{$result < 3$}{
            \Return result  \tcc{Return 0, 1, or 2 with equal probability}
        }
    }
\end{algorithm}

\noindent\textbf{3.}

The algorithm generates an $n$-bit binary number $B$ by calling $2$-WayFair($p$) $n$ times, where $n$ is chosen so that $2^n \ge k$. This $n$-bit number can represent any integer from $0$ to $2^n - 1$, each occurring with equal probability because of $2$-WayFair($p$). The algorithm accepts $B$ if it is in the range $\{0, 1, \dots, k-1\}$; otherwise, it repeats. Since each $n$-bit number has a probability of $\frac{1}{2^n}$ and there are $k$ acceptable numbers, the probability of returning any specific outcome from $0$ to $k-1$ is $\frac{1}{2^n} \times \frac{2^n}{k} = \frac{1}{k}$. Thus the algorithm returns values uniformly.

Each iteration succeeds with probability $\frac{k}{2^n}$, as there are $k$ acceptable outcomes out of $2^n$ possible $n$-bit values. Therefore, the expected number of iterations needed to generate a valid output is the reciprocal of the success probability, which gives $\frac{2^n}{k}$ iterations on average.

% Algorithm 1.3
\begin{algorithm}[H]
    \caption{k-WayFair$(p)$}
    $n \gets$ Ceil($\log_2(k)$)\tcc{Find minimum bits needed, n, such that $2^n >= k$}
    \While{true}{
        result $\gets$ 0\;
        \For {$i \gets 1$ \KwTo $n$}{
            bit $\gets$ 2-WayFair($p$)\;
            result $\gets$ 2 * result + bit\tcc{Combine bits into a number}
        }

        \If {$result < k$}{
            \Return result\tcc{Return a number between 0 and k-1} 
        }
    }
\end{algorithm}

% Problem 2
\begin{problem}
    \textbf{Uniform sample} 
\end{problem} 

\noindent\textbf{1.}

% Space complexity
\noindent\textit{Space complexity.}
The space complexity is $O(1)$, because we only need a constant amount of space to store the sample $S$.\\

% Time complexity
\noindent\textit{Time complexity.}
The per-iteration time complexity is $O(1)$.\\

% Algorithm 2.1
\begin{algorithm}[H]
    \caption{UniformSample$(stream)$}
    \tcc{Input: A stream of numbers arriving one at a time}
    \tcc{Output: A sample $S$ of 3 numbers from the stream, with each number having probability $\frac{3}{n}$ to be in $S$ after $n$ numbers have arrived}
    
    $S \gets$ an empty array of size 3\;
    $count \gets 0$ \tcc{Count of numbers seen so far}
    
    \While{stream.hasNext()}{
        $count \gets count + 1$\;
        $x \gets$ stream.next\;
        
        \uIf{$count \leq 3$}{
            $S[count - 1] \gets x$ \tcc{Directly add the first 3 numbers to $S$}
        }
        \Else{
            $r \gets$ RandomInt($1,count$)\;
            \If{$r \leq 3$}{
                $S[r - 1] \gets x$ \tcc{Replace an element in $S$}
            }
        }
    }
    
    \Return{$S$}\;
\end{algorithm}


\noindent\textbf{2.}
\noindent\textit{Proof.}
For the first three numbers in the stream, they are directly placed in the sample S, so each of these has a probability of  $\frac{3}{n}$  of remaining in S at any point.

For any subsequent number  $i$  (where $i > 3$ ), the probability of it being selected into the sample S upon arrival is  $\frac{3}{i}$ . If  $i$  is selected into S, the probability that it remains in S until the arrival of the $n_{th}$ number is given by:

$\prod_{j=i+1}^{n} \left(1 - \frac{3}{j}\right)$

After simplifying this product, it gets an overall probability of  $\frac{3}{n}$  for any specific number to be in S by the time n numbers have arrived. This ensures that each number has the same probability of being in the sample S, maintaining uniformity.

\noindent\textbf{3.}
To generalize the algorithm for maintaining a sample of size  $k$ , adjust the sample size from 3 to  $k$  and to generate a random integer  $r$  in the range  1  to  $count$ . If  $r \leq k$ , replace the  $r_{th}$ element in  S  with the current number.

\textit{Proof.}
Similarly to the case where the sample size is 3, the probability of selecting the  $i _{th}$ number is  $\frac{k}{i}$ , and the probability of it remaining in  S  is  $\frac{k}{n}$  after considering all subsequent arrivals. This approach maintains a uniform probability for each number to be in  S .

% Problem 3
\begin{problem}
    \textbf{hash} 
\end{problem} 

\noindent\textbf{1.}
% Time complexity
\textit{Time complexity.}
Calculating the hash index is $O(1)$.
Appending to the end of the linked list is $O(1)$ on average.
Thus, the average time complexity for insertion is O(1).

% Algorithm 3.1
\begin{algorithm}[H]
    \caption{Insert$(H, x, value)$}
    index $\gets$ hash(x) \% m \tcc{Compute the hash index and find the slot}
    \If{$H[index] == NULL$}{
        H[index] $\gets$ new LinkedList\tcc{Create new linked list if the slot is empty}  
    }
    H[index].add((x, value))\tcc{Add the key-value pair to the linked list}      
\end{algorithm}

\noindent\textbf{2.}
% Time complexity
\textit{Time complexity.}
Calculating the hash index is $O(1)$.
Searching the linked list for the first occurrence of x has an average time complexity of $O(\alpha)$, where $\alpha = \frac{n}{m}$ .
Thus, the average time complexity for this operation is $O(\alpha)$.

% Algorithm 3.2
\begin{algorithm}[H]
    \caption{Contains$(H, x)$}
    index $\gets$ hash(x) \% m \tcc{Compute the hash index and find the slot}
    \If{$H[index] == NULL$}{
        \Return false\tcc{Return false if the slot is empty}
    }
    \For {$(k, v)$ in H[index]}{
        \If{$k == x$}{
            \Return true\tcc{Return true if the key is found}
        }
    }
    \Return false\tcc{Return false if the key is not found}
\end{algorithm}

\noindent\textbf{3.}
Verifying that a key $y$ is not in the hash table is similar to the contain function. First compute the hash index and locate the slot. If the slot is empty, $y$ is not in the table. If the slot contains a linked list, search through it. If $y$ is not found, return False.

% Time complexity
\textit{Time complexity.}
Calculating the hash index is O(1).
In the average case, if the key $y$ is not in the table, we need to search through the linked list in that slot to confirm its absence.
The average runtime for this operation is $O(\alpha)$, similar to checking if a key is in the hash table.

\end{document}