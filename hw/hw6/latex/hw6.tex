\documentclass[11pt]{exam}
\title{\Large{\textbf{CS 630, Fall 2024, Homework 6}}}
\author{\textbf{Yifei Bao}}
\date{\textbf{November 11, 2024}}

\usepackage{fullpage}
\usepackage{xcolor}
\usepackage{amsmath,amsfonts,latexsym,amssymb}
\usepackage{enumerate}
\usepackage[hidelinks]{hyperref}
\usepackage{tikz}
\usepackage{mathtools}
\usepackage{etoolbox}
\usepackage{tipa}
\usepackage{caption}
\usepackage[boxruled,vlined,linesnumbered]{algorithm2e}
\usepackage{enumitem}

\pdfpagewidth 8.5in
\pdfpageheight 11in 

\oddsidemargin 0in
\evensidemargin 0in

\newtheorem{problem}{Problem}

\let\oldtcc\tcc
\renewcommand{\tcc}[1]{{\color{olive}\oldtcc{#1}}}

\begin{document}
\maketitle

% Problem 1
\begin{problem}
    \textbf{Bloom Filter Transformations} 
\end{problem} 

\noindent\textbf{1.}
In Bloom filter $B$, each element $x$ has $r$ hash functions that map the element to specific positions in the bit array of B, setting those bit positions to 1.

In $B'$, the original bit array is split into $B_1$ and $B_2$, and $B'$ is formed by taking the bit-wise OR of these two halves, $B' = B_1 \lor B_2$.

The new hash values in $B'$ are computed as $h_i(x) mod (m/2)$. This means that any bit that was set in either $B_1$ or $B_2$ will be folded as a 1 in $B'$ because of boolean or. Therefore, for every element originally in B, the corresponding bits will still be set in B', and the lookup will always return positive.

\noindent\textbf{2.}

\noindent\textbf{a.}
For items that were in $B$ prior to creating $B'$,
False negative is when an element that is in the set returns 0 during a lookup. Since $B'$ is created by taking the bit-wise or of $B_1$ and $B_2$, it keeps all the bit positions set by the original elements. Therefore, the false negative rate is 0.

False positive is when an element that is not in the set returns 1 during a lookup. The false positive rate of $B$ is $p = (1 - e^{-rn/m})^r$. The false positive rate of $B'$ is $p' = (1 - e^{-rn/(m/2)})^r$. Since $m/2 < m$, , the false positive rate of $B'$ is larger than that of $B$.

\noindent\textbf{b.}
For items that were not in $B$, since they were not in $B$, the false negative rate is 0 still. 

False positive rate is the same as in part a. The false positive rate of $B$ is $p = (1 - e^{-rn/m})^r$. The false positive rate of $B'$ is $p' = (1 - e^{-rn/(m/2)})^r$. Since $m/2 < m$, , the false positive rate of $B'$ is larger than that of $B$.

\noindent\textbf{c.}
$O(n)$ more items are inserted into $B'$, the false negative rate is still 0. 

The false positive rate of $B'$ is $p' = (1 - e^{-r(n+n)n/(m/2)})^r = (1 - e^{-r(2n)n/(m/2)})^r$.


% Problem 2
\begin{problem}
    \textbf{Hash Evaluations for Bloom Filters} 
\end{problem} 

\noindent\textbf{1.}
When inserting an item into the Bloom filter, the hash functions are invoked $k = \lfloor \ln 2(m/n) \rfloor$ times.

\noindent\textbf{2.}
When checking an item that was inserted into the Bloom filter, the average number of invocations of the hash functions is $k = \lfloor \ln 2(m/n) \rfloor$. Because you know the item was inserted, you can check all the hash values to see if they are set to 1.

\noindent\textbf{3.}
The probability that a bit is set to 0 after $n$ items were inserted is $p = (1 - {1/m})^{kn}$. The average number of invocations of the hash functions is $1/p$.

% Problem 3
\begin{problem}
    \textbf{hash} 
\end{problem} 

\noindent\textbf{1.}
We need at least two queries because we have two unknowns in a linear equation.

Time complexity: $O(1)$, space complexity: $O(1)$.

% Algorithm 3.1
\begin{algorithm}[H]
    \caption{Calulate$()$}
    \tcc{Call the hash function $h(x) = ax + b (mod/2^k)$ to get $h(0)$ and $h(1)$}
    $b = h(0)$\;         
    $temp = h(1)$\;      
    
    $a = (temp - b) mod 2^{64}$\;   
    
    return a, b  
\end{algorithm}

\noindent\textbf{2.}
Since $h(x)$ always results in an odd number, it is obvious that 
$(ax + b)\mod 2 = 1$, which is the same as $((a \mod 2)(x \mod 2) + (b \mod 2)) \mod 2 = 1$. When $x$ is even, then the equation becomes $(b \mod 2) \mod 2 = 1$, that $b$ is odd. When $x$ is odd, then the equation becomes $(a \mod 2) = 0$, that $a$ is even.

To avoid the problem, we can make sure that $a$ is odd, then the hash function will produce both even and odd values.

\noindent\textbf{3.}
To identify a set of keys that could be chosen to cluster in 1/1024 of the keys, we can choose $x = n*2^{k-10}$, thus $h(x) = a(n*2^{k-10}) + b \mod 2^k$. In this way, the keys are clustered in 1/1024 of the keys.

To avoid the problem, we can select a prime m close to $2^k$, and the hash function is $(ax + b)\mod m$. Using a prime modulus breaks the structure that the attacker exploits because powers of two have properties that avoid for such attacks.
\end{document}